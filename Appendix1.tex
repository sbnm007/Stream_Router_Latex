\chapter{List of Abbreviations}
% Creating a standard table with key abbreviations
\begin{table}[h!]
\centering
\begin{tabular}{p{2.5cm} p{12.5cm}}
\textbf{Abbreviation} & \textbf{Full Form} \\ \hline
API                  & Application Programming Interface \\
ConfigMap           & Kubernetes ConfigMap resource \\
CPU                 & Central Processing Unit \\
CSV                 & Comma-Separated Values \\
DNS                 & Domain Name System \\
HTTP/3              & HyperText Transfer Protocol version 3 \\
HTTPS               & HTTP over TLS \\
IP                  & Internet Protocol \\
JSON                & JavaScript Object Notation \\
JPEG                & Joint Photographic Experts Group \\
K8s                 & Kubernetes \\
LB                  & Load Balancer \\
Minikube            & Minimal single-node Kubernetes cluster \\
Pulsar              & Apache Pulsar streaming platform \\
QUIC                & Quick UDP Internet Connections \\
Secret              & Kubernetes Secret resource \\
Service             & Kubernetes Service resource \\
TCP                 & Transmission Control Protocol \\
TLS                 & Transport Layer Security \\
UDP                 & User Datagram Protocol \\
URL                 & Uniform Resource Locator \\
WebTransport        & WebTransport API over HTTP/3 \\
YAML                & YAML Ain’t Markup Language \\ \hline
\end{tabular}
\caption{List of Abbreviations}
\label{tab:abbreviations}
\end{table}

\chapter{Protocol Layering Clarification}

This appendix clarifies the relationship between \textbf{WebTransport} Streams, \textbf{HTTP/3} Streams, and \textbf{QUIC} Streams, as these terms are used interchangeably in parts of the dissertation.

\begin{itemize}
  \item \textbf{QUIC} (\emph{Quick UDP Internet Connections}) is a transport-layer protocol that operates over UDP and provides multiplexed, secure, low-latency streams.
  \item \textbf{HTTP/3} is the application-layer protocol that defines how HTTP semantics are carried inside QUIC streams.
  \item \textbf{WebTransport} is a web API that exposes both reliable (stream) and unreliable (datagram) communication channels to JavaScript clients. These channels are transported as HTTP/3 streams, which are themselves QUIC streams with additional HTTP/3 framing and control headers.
\end{itemize}


\chapter{Deployment Files}
This appendix contains the Docker files, Kubernetes manifests, configuration files, and code repository information used for deploying the custom router and microservice components of the WebTransport application.

\section{Docker Files}

This section contains the Docker files used to containerize the WebTransport router and microservice components. These Docker files define the runtime environment, dependencies, and configuration required for deploying the application components in containerized environments.

\subsection{Router Docker File}
\begin{lstlisting}[breaklines=true, caption={Router Docker File}]
FROM python:3.12-slim

# Install necessary system dependencies for Python packages and runtime
RUN apt-get update && \
    apt-get install -y --no-install-recommends \
    build-essential \
    libffi-dev \
    libssl-dev \
    wget \
    # Clean up apt cache to keep the image small
    && rm -rf /var/lib/apt/lists/*

WORKDIR /app

# Copy requirements.txt first to leverage Docker cache
COPY requirements.txt .

# Install Python dependencies
RUN pip install --no-cache-dir --upgrade pip && \
    pip install --no-cache-dir -r requirements.txt

# Copy the rest of your application code
COPY . .

RUN mkdir -p /certs /config

\end{lstlisting}

\subsection{Microservice Docker File}

\begin{lstlisting}[breaklines=true, caption={Microservice Docker File}]
FROM python:3.12-slim

# Install necessary system dependencies for Python packages and runtime
RUN apt-get update && \
    apt-get install -y --no-install-recommends \
    build-essential \
    libffi-dev \
    libssl-dev \
    wget \
    # Clean up apt cache to keep the image small
    && rm -rf /var/lib/apt/lists/*

WORKDIR /app

# Copy requirements.txt first to leverage Docker cache
COPY requirements.txt .

# Install Python dependencies
RUN pip install --no-cache-dir --upgrade pip && \
    pip install --no-cache-dir -r requirements.txt

# Copy the rest of your application code
COPY . .
\end{lstlisting}

\section{Kubernetes Manifests}

This section provides the Kubernetes deployment manifests for both the router and microservice components. These YAML files define the necessary Kubernetes resources including Deployments, Services, ConfigMaps, and Secrets required to orchestrate the WebTransport application in a Kubernetes cluster.

\subsection{Router Kubernetes Deployment}
\begin{lstlisting}[breaklines=true, caption={Router Kubernetes Deployment}, language=yaml]
# --- ConfigMap for Router Configuration ---
# This ConfigMap holds the router_config.yaml content.
# The router will load its dynamic service routing rules from here,
# enabling hot-reload of service configurations.
apiVersion: v1
kind: ConfigMap
metadata:
  name: router-config
data:
  # The key 'router_config.yaml' matches the filename expected by the router.
  # This file will be mounted into the container at /config/router_config.yaml.
  router_config.yaml: |
    # WebTransport Router Configuration
    # This file supports hot-reload - changes will be automatically detected

    global:
      default_timeout: 30
      connect_timeout: 5
      log_level: INFO

    services:
      # --- IMPORTANT: Update 'host' values for services running in K8s ---
      # For microservices deployed within Kubernetes, replace 'localhost' or
      # 'host.minikube.internal' with their Kubernetes Service Names.
      # Example: 'video' microservice exposed by 'microservice-video-service'.

      video:
        name: video
        host: microservice-video-service # Update this to your K8s Service Name
        port: 4434
        endpoint: /process_video
        content_type: application/octet-stream
        timeout: 10
        retries: 2
        data_format: binary
        enabled: true
        custom_headers:
          X-Service-Version: "1.0"
          X-Video-Quality: "hd"

      audio:
        name: audio
        host: microservice-audio-service # Update this to your K8s Service Name
        port: 4435
        endpoint: /process_audio
        content_type: application/octet-stream
        timeout: 5
        retries: 3
        data_format: binary
        enabled: true
        custom_headers:
          X-Service-Version: "1.0"
          X-Audio-Codec: "opus"

      chat:
        name: chat
        host: microservice-chat-service # Update this to your K8s Service Name
        port: 4436
        endpoint: /process_chat
        content_type: application/json
        timeout: 3
        retries: 1
        data_format: json
        enabled: true
        custom_headers:
          X-Service-Version: "1.0"
          X-Chat-Mode: "realtime"

      # --- Other services (examples) ---
      # For services not managed by Kubernetes (e.g., running on your host machine
      # or external services), you might use 'host.minikube.internal' or their external IP.
      screen:
        name: screen
        host: host.minikube.internal # Example: if screen service runs on your host
        port: 4437
        endpoint: /process_screen
        content_type: application/octet-stream
        timeout: 8
        retries: 2
        data_format: binary
        enabled: false
        custom_headers:
          X-Service-Version: "1.0"
          X-Screen-Mode: "shared"

      analytics:
        name: analytics
        host: analytics-server-service # Example: if analytics is another K8s service
        port: 9090
        endpoint: /track_event
        content_type: application/json
        timeout: 2
        retries: 1
        data_format: json
        enabled: false
        custom_headers:
          X-Service-Version: "2.0"
          X-Analytics-Type: "stream"

---

# --- Deployment for WebTransport Router ---
# This Deployment manages the router application pods, ensuring desired replica count
# and automatic restarts in case of failures.
apiVersion: apps/v1
kind: Deployment
metadata:
  name: webtransport-router-deployment
  labels:
    app: webtransport-router
spec:
  replicas: 1 # Adjust replica count for high availability/load balancing
  selector:
    matchLabels:
      app: webtransport-router
  template:
    metadata:
      labels:
        app: webtransport-router
    spec:
      containers:
        - name: router
          image: sbnm007/quic_router:0.0.2 # Replace with your built Docker image
          # Define the command and arguments to run the router application.
          # This overrides any CMD instruction in the Dockerfile.
          command: ["python"]
          args: ["router.py"]
          ports:
            - containerPort: 4433 # Internal container port for QUIC/HTTP/3 (TCP fallback)
              name: quic-h3-tcp
              protocol: TCP
            - containerPort: 4433 # Internal container port for QUIC (UDP)
              name: quic-h3-udp
              protocol: UDP
          volumeMounts:
            # Mount the router_config.yaml from the 'router-config' ConfigMap.
            # It will appear as a file at /config/router_config.yaml inside the container.
            - name: router-config-volume
              mountPath: /config/router_config.yaml
              subPath: router_config.yaml
            # Mount the TLS certificates from the existing 'quic-cert' Secret.
            # The 'tls.crt' and 'tls.key' from the secret are mapped to 'new-quic.crt'
            # and 'new-quic.key' respectively, at /certs/ inside the container.
            - name: router-certs-volume
              mountPath: /certs/ # Mount the entire secret as a directory
          env: # Define environment variables for router.py to find the mounted files
            - name: CERT_PATH
              value: /certs/new-quic.crt # CORRECTED: Use the actual mounted filename
            - name: KEY_PATH
              value: /certs/new-quic.key # CORRECTED: Use the actual mounted filename
            - name: CONFIG_PATH
              value: /config/router_config.yaml
      volumes:
        - name: router-config-volume
          configMap:
            name: router-config # Refers to the ConfigMap defined above
        - name: router-certs-volume
          secret:
            secretName: quic-cert # Reusing your existing 'quic-cert' Secret
            items: # Explicitly map secret keys to desired filenames in the volume
              - key: tls.crt
                path: new-quic.crt # This means the file will be /certs/new-quic.crt
              - key: tls.key
                path: new-quic.key # This means the file will be /certs/new-quic.key

---

# --- Service for WebTransquicport Router ---
# This Service exposes the router application.
# - 'LoadBalancer' type is used to expose it externally (e.g., Minikube will provide a host port).
# - 'ClusterIP' can be used if access is only needed from within the cluster.
apiVersion: v1
kind: Service
metadata:
  name: webtransport-router-service
spec:
  selector:
    app: webtransport-router # Selects pods managed by the router deployment
  ports:
    - name: https-quic-tcp # Service port for HTTP/3 over TCP fallback
      protocol: TCP
      port: 443 # External port clients will connect to
      targetPort: 4433 # Internal container port
    - name: https-quic-udp # Service port for QUIC (HTTP/3) over UDP
      protocol: UDP
      port: 443 # External port clients will connect to
      targetPort: 4433 # Internal container port
  type: LoadBalancer # Use LoadBalancer for external access (Minikube maps to host port)
  # For bare metal with MetalLB, LoadBalancer will acquire an external IP.
  # If you only need internal access, use 'ClusterIP'.

\end{lstlisting}


\subsection{Microservice Kubernetes Deployments}
\begin{lstlisting}[breaklines=true, caption={Microservice Kubernetes Deployments}, language=yaml]
# --- ConfigMap for Shared Application Configuration (for Microservices) ---
# This ConfigMap holds the microservice_config.yaml content, which will be
# mounted as a file into each microservice pod. The microservice will
# hot-reload its settings from this file.
apiVersion: v1
kind: ConfigMap
metadata:
  name: microservice-app-config
data:
  # The key 'microservice_config.yaml' matches the filename we'll mount.
  microservice_config.yaml: |
    # Microservice-specific configuration
    # CORRECTED: Pointed to the correct internal service name for the broker.
    pulsar_broker_url: "pulsar://10.100.48.76:6650"
    send_to_pulsar: true 

---

# --- Deployment for Video Microservice ---
# Manages the 'video' microservice pods.
apiVersion: apps/v1
kind: Deployment
metadata:
  name: microservice-video-deployment
  labels:
    app: video
spec:
  replicas: 1
  selector:
    matchLabels:
      app: video
  template:
    metadata:
      labels:
        type: microservice
        app: video
    spec:
      containers:
        - name: video-server
          image: sbnm007/microservice:0.0.6 # Ensure this is your rebuilt image
          command: ["python"]
          args:
            - "microservice.py"
            - "video"
            - "4434"
          ports:
            - containerPort: 4434
              protocol: TCP
          volumeMounts:
            # Mount the microservice_config.yaml from the ConfigMap
            # It will appear as a file at /config/microservice_config.yaml inside the container.
            - name: microservice-config-volume
              mountPath: /config
          env: # Pass the path to the config file as an environment variable
            - name: MICROSERVICE_CONFIG_PATH
              value: /config/microservice_config.yaml
      volumes:
        - name: microservice-config-volume
          configMap:
            name: microservice-app-config # Refers to the ConfigMap defined above

---

# --- Service for Video Microservice ---
apiVersion: v1
kind: Service
metadata:
  name: microservice-video-service
spec:
  selector:
    app: video
  ports:
    - protocol: TCP
      port: 4434
      targetPort: 4434
  type: ClusterIP

---

# --- Deployment for Audio Microservice ---
apiVersion: apps/v1
kind: Deployment
metadata:
  name: microservice-audio-deployment
  labels:
    app: audio
spec:
  replicas: 1
  selector:
    matchLabels:
      app: audio
  template:
    metadata:
      labels:
        type: microservice
        app: audio
    spec:
      containers:
        - name: audio-server
          image: sbnm007/microservice:0.0.6 # Ensure this is your rebuilt image
          command: ["python"]
          args:
            - "microservice.py"
            - "audio"
            - "4435"
          ports:
            - containerPort: 4435
              protocol: TCP
          volumeMounts:
            - name: microservice-config-volume
              mountPath: /config
          env:
            - name: MICROSERVICE_CONFIG_PATH
              value: /config/microservice_config.yaml
      volumes:
        - name: microservice-config-volume
          configMap:
            name: microservice-app-config

---

# --- Service for Audio Microservice ---
apiVersion: v1
kind: Service
metadata:
  name: microservice-audio-service
spec:
  selector:
    app: audio
  ports:
    - protocol: TCP
      port: 4435
      targetPort: 4435
  type: ClusterIP

---

# --- Deployment for Chat Microservice ---
apiVersion: apps/v1
kind: Deployment
metadata:
  name: microservice-chat-deployment
  labels:
    app: chat
spec:
  replicas: 1
  selector:
    matchLabels:
      app: chat
  template:
    metadata:
      labels:
        type: microservice
        app: chat
    spec:
      containers:
        - name: chat-server
          image: sbnm007/microservice:0.0.6 # Ensure this is your rebuilt image
          command: ["python"]
          args:
            - "microservice.py"
            - "chat"
            - "4436"
          ports:
            - containerPort: 4436
              protocol: TCP
          volumeMounts:
            - name: microservice-config-volume
              mountPath: /config
          env:
            - name: MICROSERVICE_CONFIG_PATH
              value: /config/microservice_config.yaml
      volumes:
        - name: microservice-config-volume
          configMap:
            name: microservice-app-config

---

# --- Service for Chat Microservice ---
apiVersion: v1
kind: Service
metadata:
  name: microservice-chat-service
spec:
  selector:
    app: chat
  ports:
    - protocol: TCP
      port: 4436
      targetPort: 4436
  type: ClusterIP

\end{lstlisting}

\section{Configuration Files}

\begin{lstlisting}[breaklines=true, caption={Router Configuration File}, language=yaml]
# WebTransport Router Configuration
# This file supports hot-reload - changes will be automatically detected

global:
  default_timeout: 30
  connect_timeout: 5
  log_level: INFO

services:
  # Existing services
  video:
    name: video
    host: localhost
    port: 4434
    endpoint: /process_video
    content_type: application/octet-stream
    timeout: 10
    retries: 2
    data_format: binary
    enabled: true
    custom_headers:
      X-Service-Version: "1.0"
      X-Video-Quality: "hd"

  audio:
    name: audio
    host: localhost
    port: 4435
    endpoint: /process_audio
    content_type: application/octet-stream
    timeout: 5
    retries: 3
    data_format: binary
    enabled: true
    custom_headers:
      X-Service-Version: "1.0"
      X-Audio-Codec: "opus"

  chat:
    name: chat
    host: localhost
    port: 4436
    endpoint: /process_chat
    content_type: application/json
    timeout: 3
    retries: 1
    data_format: json
    enabled: true
    custom_headers:
      X-Service-Version: "1.0"
      X-Chat-Mode: "realtime"

  # NEW SERVICES - Just add them here and they work automatically!
  
  screen:
    name: screen
    host: localhost
    port: 4437
    endpoint: /process_screen
    content_type: application/octet-stream
    timeout: 8
    retries: 2
    data_format: binary
    enabled: true  # Set to true to enable  
    custom_headers:
      X-Service-Version: "1.0"
      X-Screen-Mode: "shared"
\end{lstlisting}

\section{Code Repository}
The complete source code for this WebTransport router implementation is publicly available on GitHub for reproducibility and future research. The repository includes the WebTransport client implementation, custom QUIC router with HTTP/3 support, microservice implementations for processing different media types, Kubernetes deployment manifests, and comprehensive documentation for setup and deployment. The repository can be accessed at \url{https://github.com/sbnm007/webtransport-quic-router}.


% \chapter{Writing Guidelines}

% The intention of this document is two-fold: To provide a LaTeX template that facilitates the writing of a dissertation in LaTeX and to give guidance in writing a dissertation. As such, it attempts to serve two masters at once which is generally not a good idea. In order to address this, the content of this document has been split into the discussion of the structure of a dissertation document and the content that focusses on facilitating the writing of a dissertation using LaTeX.

% The following section are meant as general guidelines for the writing of content for a dissertation and to provide guidance for the use of LaTeX features in order to avoid spending significant amount of time on tweaking LaTeX features to include content e.g. it is easy to waste significant amount of time on the inclusion of diagrams, especially for new users of LaTeX. The sections below are an attempt to avoid this waste of time and to allow students to concentrate on creating content instead of diving into LaTeX rabbitholes.

% Read the tex sources i.e. the thesis.tex file and adapt the variables to suit your case. This file aslo includes some macros that facilitate the writing of the dissertation as LaTeX document.


% \section{Style of English}
% \label{sec:StyleOfEnglish}
% %%TODO Revise text from 2000-version of project information

% An impersonal style keeps the technical factors and ideas to the forefront of the discussion and you in the background. Try to be objective and quantitative in your conclusions. For example, it is not enough to say vaguely “because the compiler was unreliable the code produced was not adequate”. It would be much better to say “because the XYZ compiler produced code which ran 2-3 times slower than PQR (see Table x,y), a fast enough scheduler could not be written using this algorithm”. The second version gives the reader with definite statements and conveys a detailed understanding of the subject.

% The following points are couple of {\it Do's \& Dont's} that I have noted down as feedback to reports over the years. The focus of this list is to encourage writers to be specific in writing reports - some of this is motivated by Strunk and White's The Elements of Style~(\cite{strunk}). Regarding reports that are submitted as part of a degree, examiners have to read and mark these reports - make it easy for these examiners to give good marks by following a number of simple points:

% \begin{description}
% 	\item [Titles:] No title should be immediately followed by another title i.e. whenever there are two titles without text between them, it is an indication that something is missing e.g. a chapter should always start with an introduction that explains what will be discussed in the chapter, what sections it includes and how these sections hang together. The title of a chapter should never be immediately followed by the title of the first section. 
% 	\item [Figures:] Figures and graphs should have sufficient resolution; figures with low resolution appear blurred and require the reader to make assumptions.
% 	\item  [Captions:] Use captions to describe a figure or table to the reader. The reader should not be forced to search through text to find a description of a figure or table. If you do not provide an interpretation of a figure or table, the reader will make up their own interpretation and given Murphy's law, will arrive at the polar opposite of what was intended by the figure or table.
% 	\item [Backgrounds:] Backgrounds of figures and snapshots of screens should be light. Developers often use terminals or development environments with dark backgrounds. Snapshots of these terminals or developments are difficult to read when placed into a report. 
% 	\item [Acronyms:] Acronyms should be introduced by the words they represent followed by the acronym in capitals enclosed in brackets e.g. "...TCP (Transmission Control Protocol)..." $\Rightarrow$  "... Transmission Control Protocol (TCP)..."
% 	\item [Contractions:] I would generally suggest to avoid contractions such as "I'd", "They've", etc in reports. In some cases, they are ambiguous e.g. "I'd" $\Rightarrow$ "I would" or "I had" and can lead to misunderstandings.
% 	\item [Avoid "do":] Be specific and use specific verbs to describe actions.
% 	\item [Adverbs:] Adverbs and adjectives such as "easily", "generally", etc should be removed because they are unspecific e.g. the statement "can be easily implemented" depends very much on the developer. 
% 	\item [Articles:] "A" and "an" are indefinite articles; they should be used if the subject is unknown. "The" is a definite article; which should be used if a specific subject is referred to. For example, the subject referred to in "allocated by the coordinator" is not determined at the time of writing and so the sentences should be changed to "allocated by a coordinator".
% 	\item [Avoid brackets:] Brackets should not be used to hide sub-sentences, examples or alternatives. The problem with this use of brackets is that it is not specific and keeps the reader guessing the exact meaning that is intended. For example "... system entities (users, networks and services) through ..." should be replaced by "... system entities such as users, networks, and services through ...".
% 	\item [Punctuation:] A statement is concluded with a period; a question with a question mark.  
% 	\item [Spellcheckers:] Use a spellchecker!
% \end{description}


% \section{Figures} 
% \label{app:figures}

% The arranging of figures in Latex can lead to spending a lot of time on minor issues e.g. positioning a figure in a specific location on a page, fixing minor issues with an exact size of a figure, etc. Figure~\ref{fig:ImageOfAChick} provides a simple example that demonstrates the use of one of two macros for handling figures {\it includescalefigure}. Figures should always be readable without magnification when printed and the resolution of an image should be sufficient to provide a clear picture when printed.

% \includescalefigure{fig:ImageOfAChick}{An Image of a chick}{A caption should describe the figure to the reader and explain to the reader the meaning of the figure. A Sub-clause of Murphy's Law: If the interpretation of a figure is left to a reader, the reader will misinterpret the figure, feel insulted or decide to ignore it. Do not leave it up to the reader!}{0.2}{image.png}

% \begin{verbatim}
%     \includescalefigure{fig:ImageOfAChick}{An Image of a chick}
%     {A caption should describe the figure to the reader and 
%     explain to the reader the meaning of the figure. A Sub-
%     clause of Murphy's Law: If the interpretation of a figure
%     is left to a reader, the reader will misinterpret the 
%     figure, feel insulted or decide to ignore it. Do not 
%     leave it up to the reader!}{0.2}{image.png}    
% \end{verbatim}


% \section{Code Snippets}

% The following are two examples of how to incldue code snippets in your dissertation. - A dissertation or report should never include complete copies of an implementation; complete copies should be provided through online repositories or storage devices e.g. CDs, USB sticks, etc. 

% The first example, listing~\ref{lst:snippet} demonstrates the use of macro that includes a code snippet from a file, snippet.py e.g. 

% \begin{center}
%     \begin{BVerbatim}

% \includecode{Sample Code}{Lengthy caption explaining 
% the code to the reader}{lst:snippet}{snippet.py}.

%     \end{BVerbatim}
% \end{center}

% %% Short caption for the table of listings - long caption for the explanation for the reader
% \includecode{Sample Code}{Lengthy caption explaining the code to the reader}{lst:snippet}{snippet.py}


% The second example, listing~\ref{lst:snippet2}, uses the basic support for listings in LaTeX, where the code is directly included in the LaTeX file.

% %% Short caption for the table of listings - long caption for the explanation for the reader
% \begin{lstlisting}[caption={[Sample Code 2]Second Lengthy caption}, label={lst:snippet2}]
% x = 1
% if x == 1:
%     # indented four spaces
%     print("x is 1.")
% \end{lstlisting}


% %%
% %%
% \section{Tables}
% \label{app:tables}

% The table below, table~\ref{tab:SummaryProjects}, is a simplistic example of a table in LaTeX. When writing your dissertation, you can copy the content from the appendix file into the location in your document and adapt it, or use a tool to design your table.

% \begin{table}[!ht]
%     \begin{center}
%         \begin{tabular}{|l|p{3cm}|p{3cm}|} 
%         \hline
%          \bf  & \bf Aspect \#1  & \bf Aspect \#2 \\
%           \hline
%         Row 1 & Item 1 & Item 2 \\
%         Row 2 & Item 1 & Item 2 \\
%         Row 3 & Item 1 & Item 2 \\
%         Row 4 & Item 1 & Item 2 \\
%         \hline
%         \end{tabular} 
%     \end{center}     
%     \caption[Comparison of Closely-Related Projects]{Caption that explains the table to the reader}	
%     \label{tab:SummaryProjects}
%     \end{table}

% The table~\ref{tab:ScaleSummaryProjects} demonstrates adapting the size of tables to the width of a document.

%     \begin{table}[!ht]
%         \begin{center}
%             \begin{tabularx}{0.9\textwidth}{ 
%                 | >{\raggedright\arraybackslash}p{2cm} 
%                 | >{\centering\arraybackslash}X 
%                 | >{\centering\arraybackslash}X | }
%                 \hline
%              \bf  & \bf Aspect \#1  & \bf Aspect \#2  \\
%              \hline
%             Row 1 & Item 1 & Item 2 \\
%             Row 2 & Item 1 & Item 2 \\
%             Row 3 & Item 1 & Item 2 \\
%             Row 4 & Item 1 & Item 2 \\
%             \hline
%             \end{tabularx} 
%         \end{center}     
%         \caption[Salable Table]{Example of a scalable table with the first column being fixed to a very specific size and the two following columns splitting the remaining width between them and these columns being centered while the first column is aligned to the rigth.}	
%         \label{tab:ScaleSummaryProjects}
%         \end{table}
    

% \begin{table}[!ht]
% \begin{center}
% 	\begin{tabular}{|l|l|c|} 
% 	\hline
%  	\bf Parameter 1  & \bf Parameter 2  & \bf Measurements  \\
%   	\hline
% 	$Option_1,Option_2$ & $P_{11}, P_{12}$ & ? \\
% 	$Option_2,Option_3$ & $P_{21}, P_{22}, P_{33}$ &  ? \\
% 	$Option_3$ & $P_{31}, P_{32}$ & ? \\
% 	$Option_4$ & $P_{41}, P_{42}, P_{43}$ & ? \\
% 	\hline
% 	\end{tabular}
% \end{center}
% \caption[Variables of the experiment]{Evaluation Matrix for experiment that lists the possible parameters that can be varied and the measurements that could be made for each experiment}	
% \label{tab:experimentsetup}
% \end{table}

